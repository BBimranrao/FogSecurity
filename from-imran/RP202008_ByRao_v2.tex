%\documentclass[10pt][runningheads]{llncs}
\documentclass[10pt]{llncs}
%\documentclass[transaction]{IEEEtran}
% IEEETran already defines natbib package. undefined it so that we can use a more customized natbib
\makeatletter
\let\NAT@parse\undefined
\makeatother
%% INFOCOM 2010 addition: 
\makeatletter
\def\ps@headings{%
\def\@oddhead{\mbox{}\scriptsize\rightmark \hfil \thepage}%
\def\@evenhead{\scriptsize\thepage \hfil \leftmark\mbox{}}%
\def\@oddfoot{}%
\def\@evenfoot{}}
\makeatother
%\pagestyle{headings}
%Furthermore change the line \pagestyle{headings} to \pagestyle{empty}
\usepackage[pdftex]{graphicx}
\usepackage{subfigure}
%\usepackage[figure,boxed]{algorithm2e} % Formats Algorithms as float objects
\usepackage{algorithm2e} % Formats Algorithms as float objects
\usepackage{epsfig}
\usepackage{fullpage}
\usepackage{setspace}
\usepackage{verbatim} 
\usepackage{amsfonts}
\usepackage{tabularx}
\usepackage{booktabs}
\usepackage[square, comma, sort, numbers]{natbib}
%
%============================================================
\makeatletter
\renewcommand\bibsection%
{
  \section*{\refname
    \@mkboth{\MakeUppercase{\refname}}{\MakeUppercase{\refname}}}
}
\makeatother
%============================================================
\newenvironment{algorithmfloating}{%
\renewenvironment{algocf}[1][h]{}{}% pass over the floating stuff
\algorithm
}{%
\endalgorithm
}
%============================================================
%
%\doublespacing
%\setlength{\parskip}{12pt} % 12 pt = space between paragraphs
%\setlength{\parindent}{3pt} % 0 pt = indentation
\begin{document}
%\pagestyle{headings}
%\pagestyle{empty}
%
%============================================================
\title{Research Proposal}
%
%============================================================
\author{
Nomica Choudhry (\selectfont\ttfamily\upshape nomica19@yahoo.com)\\
}
%abbreviated author list (for running head)
%\authorrunning{Imran Rao et al.}
\institute{
\fontsize{12}{12}\selectfont\itshape
{\bf Security Threat Identification for Big Data in Smart Cities}\\
}
\maketitle

%
%==========================
\section{Introduction}
International Organization for Migration (Bret 2015) estimated that around 3 million people are moving to cities every week and by 2040 it is estimated that around 65\% population will be living in cities for better life style. Managing such a massive population by traditional approaches to city management without the intervention of technology is an interesting area of research. Governments at different levels - regional, national, and international have initiated programs on digital and smart cities.

To make the concept of smart city a reality, Internet of Things (IoT) has gained significance attention (Vlacheas et al.). IoT is a network where wired and wireless devices; such as RFID (radio-frequency identification) systems, embedded sensors and actuator nodes, are inter-connected and communicate with each enabling smart and autonomous services. These services help solve the problems related to traffic management, climate change, urban planning, security and surveillance – to name a few. These IOT services utilize cloud based protocols and technologies to accumulate data from wireless and wired sensing devices. Using artificial intelligence techniques, this data is further analysed to extract patterns and recognized events. However, even though it is being considered that IoT applications go very well with cloud computing, there are still major issues hindering to make it a reality including mobility support, geo-distribution, location-awareness and low latency. 

The term fog computing coined by CISCO in 2011 has undergone rapid progress in the last decade. Fog computing facilitates efficient communication, computation and storage to IOT devices or fog nodes, as they become more powerful in capacities and functionalities. Fog computing is linked to new access control model to lessen the overhead costs by moving the execution of application logic from the centre of the cloud data sources to the periphery of the IoT-oriented sensor networks. As an extension of cloud computing, fog computing can solve problems such as high latency, lack of support for mobility and location awareness in cloud computing. A series of IoT devices can be connected to the fog nodes that assist a cloud service center to store and process a part of data in advance. Not only can it reduce the pressure of processing data, but also improve the real-time and service quality. However, data processing at fog nodes suffers from many challenging issues, such as false data injection attacks, data modification attacks, and IoT devices’ privacy violation.

Fog computing provides a lot of benefits over traditional cloud computing. The main advantage of fog computing over cloud computing is that cloud is a centralized system, while the fog is a distributed decentralized infrastructure. In case of cloud computing, the amount of data transmitted over the network is more. It results in higher probability of error as bit error, data transmission latency and packet dropping possibility are proportional to the volume of transmitted data. 

At the same times, fog computing also facilitate large scale data by providing elastic resources to it without having any issue with high latency which is again a major draw back of cloud computing. If fog computing ic coupled with cloud computing, it can handle big data, its aggregation and pre-processing which results in reduced data transfer and storage resulting in efficient use of resources. We believe that by using fog technology and do much of the data processing at the fog end by using an efficient aggregation algorithm will tackle large scale data generated by IoTs in a very efficient way.

The most important goal of fog computing is to improve efficiency and reduce the amount of data that needs to be transported to the cloud for processing, analysis and storage. Although this is mostly done for efficiency reasons, it can also be done for security and compliance reasons. At the same times as the distance between the user and cloud is large it results in more time to transmit packages between them which results in lower quality of service, bad user experience and weak security and reliability. Fog computing, on the other hand provides a lot of benefits over traditional cloud computing including low latency, quick response, location awareness, better reliability, enhanced security along with fog volatility. In short, cloud and fog computing both offer end users data, storage, computation and application services, but fog computing is in much closer proximity to end users and better supports mobility. To be clear, fog computing will not replace cloud computing altogether; rather, it's a supplement to the cloud.

Fog computing can handle massive data which arises from the IOT (Internet of Things) on the edge of the network. Because of its characteristics like low latency, mobility, heterogeneity it is considered to be the best platform for IOT. Because the fog nodes are deployed at the edge of the network and low-traffic nodes, they are more vulnerable to hackers. 

As we are living in an interconnected IOT environment where data is coming from different  sources we need to pay attention to data breach cases and privacy threats. Moreover, in IoT, in order to reduce communication costs, it is essential to aggregate individual IoT device’s data at associated fog device.\\

\textbf{Authentication and Trust issues:} Authentication, being an important issue of fog computing, can not be ignored. This is because, these days when we talk about data we mean huge or large scale of it. The trust issue arises in fog computing as theses serives are being provided by different service providers including cloud service provider, internet service provider and end users. A rouge fog node is a fog device which pass one self of as legal and persuade end user to connect to it. Once connected, it manipulates the signals coming to and from the user to the cloud and can easily launch attacks.\\

\textbf{Privacy:} Privacy concern is a genuine problem in fog computing as there are many networks involved and it is based on wireless technology making it more prone to attacks. As fog nodes is easily acessible to the end users which can easily access information from it and can inject it with malicious data.\\

\textbf{Security:} Security can esaily be compromised in fog computing there are number of devices connected to fog nodes and at different gateways. Each device has a different IP address, and any hacker can fake your IP address to gain access to your personal information that is stored in that particular fog node.\\

\textbf{Fog Servers:} Right placement of fog servers should be there so that it can deliver its maximum service. The company should analyze the demand and work done by the fog node before placing it will help in reducing the maintenance cost.\\

\textbf{Energy consumption:} One of the major issue with fog computing is that it is high in energy consumption. \\

However, without explicit human intervention, the data and events accumulated by the IoT devices might not be further used for any meaningful purposes. We need this sensed data to be fused with the context it has been accumulated from for an autonomous and self-sustained smart applications.

%
%----------------------------------------------------
\subsection{Background}



%
% ----------
% General Security Protocols for Fog Computing

%
% ----------
% AI based Security Protocols for Fog Computing

%
% ----------
% NN based Security Protocols for Fog Computing

%
% ----------
% Deep Learning based Security Protocols for Fog Computing


%
%----------------------------------------------------
\subsection{Research Issues}


%
% ----------
% Research Issue # 1 in Deep Learning based Security Protocols for Fog Computing


%
% ----------
% Research Issue # 2 in Deep Learning based Security Protocols for Fog Computing


%
% ----------
% Research Issue # 3 in Deep Learning based Security Protocols for Fog Computing


%
%----------------------------------------------------
\subsection{Research Problem, Motivation and Objectives}

%
% ----------
\subsubsection{Research Probem}

% Explain the research problem we are addressing

%
% ----------
\subsubsection{Motivation}

% Expalin Why we are addressing this research problem

%
% ----------
% Why are we researching a security protocol?

%
% ----------
% Why are we researching an Deep Learning based security protocol?

%
% ----------
% Why are we researching a regression protocol?

%
% ----------
\subsubsection{Applications}
Recently, the adoption of fog and IoT in the healthcare field has been significantly improved health services and contributed to its innovation. Health monitoring systems have been using remote cloud servers for storing and processing huge amount of data coming from large amount of sensors. These systems however suffer from a lot of issues related to latency, location awareness and redundant data. These issues can not be tolerated in healthcare system as single error in analyzing data causes inaccurate treatment decisions and can cause precious human life. A better solution is to provide an extra layer in between a conventional gateway and a remote cloud server. The extra layer denoted as fog layer helps diminishing the volume of transmitted data for guaranteeing. QoS, and saving network bandwidth by preprocessing data. At the same times, fog computing offers advanced services at the edge of the network and reduces the burden of cloud.

%
% ----------
\subsubsection{Research Objective}

% Explain the purpose of this research and the objectives of this study


%
%----------------------------------------------------
\subsection{The Contributions}


%
%----------------------------------------------------
\subsection{Thesis Organization}


%
%==========================
\section{Survey}

% Overview of the Research in the field and Taxonomy

%
%----------------------------------------------------
\subsection{Survey 1}

% Brief Overview:

% Pros

% Cons

% Comparative Analysis with our work


%
%----------------------------------------------------
\subsection{Survey 2}

% Brief Overview:

% Pros

% Cons

% Comparative Analysis with our work


%
%----------------------------------------------------
\subsection{Survey 3}

% Brief Overview:

% Pros

% Cons

% Comparative Analysis with our work


%
%----------------------------------------------------
\subsection{Survey 3}

% Brief Overview:

% Pros

% Cons

% Comparative Analysis with our work


%
%----------------------------------------------------
\subsection{Summary of the Surveyed Research Work}



%
%==========================
\section{System Model and Evaluation Metrics}

%
%----------------------------------------------------
\subsection{Scope of the Research}

% In this section, we describe the scope and limitations of the research

%
%----------------------------------------------------
\subsection{Frequently Used Terms}

%
%----------------------------------------------------
\subsection{System Model}

% Articulate the methamatical model of the system 

%
%----------------------------------------------------
\subsection{Evaluation Metrics}

% Define terms like accuracy, error, convergence, correctness, etc.

%
%----------------------------------------------------
\subsection{Simulation Model}

% Define simulation parameteres and asumptions made

%
%----------------------------------------------------
\subsection{Simulator Design Architecture}


%
%----------------------------------------------------
\subsection{Simulation Inputs and Desired Output}


%
%==========================
\section{Proposed Protocol}

%
%----------------------------------------------------
\subsection{Describe the Protocol}

%
%----------------------------------------------------
\subsection{Present the Pseudo Code of the Protocol}

%
%----------------------------------------------------
\subsection{Describe the State Diagram of the Protocol}

%
%----------------------------------------------------
\subsection{Describe the Activity Diagram of the Protocol}


%
%----------------------------------------------------
\subsection{Critically analyze the Protocol}

% Describe the Pros, Cons and the Future work of the propsoed protocol


%
%==========================
\section{Simulation and Results}


%
%==========================
\section{Conclusion}
 
%
%==========================

%\bibliographystyle{IEEEtran}
\bibliographystyle{splncs}
\footnotesize{
%\bibliographystyle{plainnat}
%\bibliography{RP2016_MIS_v1}
}
%============================================================
%
\end{document}