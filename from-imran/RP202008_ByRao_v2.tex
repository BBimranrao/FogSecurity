%\documentclass[10pt][runningheads]{llncs}
\documentclass[10pt]{llncs}
%\documentclass[transaction]{IEEEtran}
% IEEETran already defines natbib package. undefined it so that we can use a more customized natbib
\makeatletter
\let\NAT@parse\undefined
\makeatother
%% INFOCOM 2010 addition: 
\makeatletter
\def\ps@headings{%
\def\@oddhead{\mbox{}\scriptsize\rightmark \hfil \thepage}%
\def\@evenhead{\scriptsize\thepage \hfil \leftmark\mbox{}}%
\def\@oddfoot{}%
\def\@evenfoot{}}
\makeatother
%\pagestyle{headings}
%Furthermore change the line \pagestyle{headings} to \pagestyle{empty}
\usepackage[pdftex]{graphicx}
\usepackage{subfigure}
%\usepackage[figure,boxed]{algorithm2e} % Formats Algorithms as float objects
\usepackage{algorithm2e} % Formats Algorithms as float objects
\usepackage{epsfig}
\usepackage{fullpage}
\usepackage{setspace}
\usepackage{verbatim} 
\usepackage{amsfonts}
\usepackage{tabularx}
\usepackage{booktabs}
\usepackage[square, comma, sort, numbers]{natbib}
%
%============================================================
\makeatletter
\renewcommand\bibsection%
{
  \section*{\refname
    \@mkboth{\MakeUppercase{\refname}}{\MakeUppercase{\refname}}}
}
\makeatother
%============================================================
\newenvironment{algorithmfloating}{%
\renewenvironment{algocf}[1][h]{}{}% pass over the floating stuff
\algorithm
}{%
\endalgorithm
}
%============================================================
%
%\doublespacing
%\setlength{\parskip}{12pt} % 12 pt = space between paragraphs
%\setlength{\parindent}{3pt} % 0 pt = indentation
\begin{document}
%\pagestyle{headings}
%\pagestyle{empty}
%
%============================================================
\title{Research Proposal}
%
%============================================================
\author{
Nomica Choudhry (\selectfont\ttfamily\upshape nomica19@yahoo.com)\\
}
%abbreviated author list (for running head)
%\authorrunning{Imran Rao et al.}
\institute{
\fontsize{12}{12}\selectfont\itshape
{\bf An interaction index based Deep Learning framework for predictive and preventive measures for IoTs}\\
}
\maketitle

%
%==========================
\section{Introduction}
International Organization for Migration (Bret 2015) estimated that around 3 million people are moving to cities every week and by 2040 it is estimated that around 65\% population will be living in cities for better life style. Managing such a massive population and infrastructre with the intervention of technology is an interesting area of research and governments at different levels - regional, national, and international have initiated programs on digital and smart cities.

To make the concept of smart city a reality, Internet of Things (IoT) has gained significance attention (Vlacheas et al.). IoT is a network where wired and wireless devices; such as RFID (radio-frequency identification) systems, embedded sensors and actuator nodes, are inter-connected and communicate with each enabling smart and autonomous services. These services help solve the problems related to traffic management, climate change, urban planning, security and surveillance – to name a few. 

%
%----------------------------------------------------
\subsection{Background}
These IOT services utilize cloud based protocols and technologies to accumulate data from wireless and wired sensing devices. As this data is coming from different sources we need to pay attention to data breach cases and privacy threats. Security can esaily be compromised in IoT computing there are number of devices connected via different gateways. Each device has a different IP address, and any hacker can fake the IP address to gain access to the personal information that is stored in that particular IoT device.

Authentication, being an important issue of IoT computing, can not be ignored. A rouge IoT device is a device which passes oneself of as legal and persuade end user to connect to it. Once connected, it manipulates the signals coming to and from the user to the cloud and can easily launch attacks. Privacy concern is also a genuine problem in IoT computing as there are many networks involved and it is based on wireless technology making it more prone to malicious attacks.

Moreover, due to the very large scale of the data sensed by the IoT devices, the security of the data accumulated with human intervention is not feasible. We need this sensed data to be secured an autonomous and self-sustained manner using artificial intelligence techniques. 

%
% ----------
% General Security Protocols for Fog Computing

%
% ----------
% AI based Security Protocols for Fog Computing

%
% ----------
% NN based Security Protocols for Fog Computing

%
% ----------
% Deep Learning based Security Protocols for Fog Computing


%
%----------------------------------------------------
\subsection{Research Issues}


%
% ----------
% Research Issue # 1 in Deep Learning based Security Protocols for Fog Computing


%
% ----------
% Research Issue # 2 in Deep Learning based Security Protocols for Fog Computing


%
% ----------
% Research Issue # 3 in Deep Learning based Security Protocols for Fog Computing


%
%----------------------------------------------------
\subsection{Research Problem, Motivation and Objectives}

%
% ----------
\subsubsection{Research Probem}

% Explain the research problem we are addressing

%
% ----------
\subsubsection{Motivation}

% Expalin Why we are addressing this research problem

%
% ----------
% Why are we researching a security protocol?

%
% ----------
% Why are we researching an Deep Learning based security protocol?

%
% ----------
% Why are we researching a regression protocol?

%
% ----------
\subsubsection{Applications}
Recently, the adoption of fog and IoT in the healthcare field has been significantly improved health services and contributed to its innovation. Health monitoring systems have been using remote cloud servers for storing and processing huge amount of data coming from large amount of sensors. These systems however suffer from a lot of issues related to latency, location awareness and redundant data. These issues can not be tolerated in healthcare system as single error in analyzing data causes inaccurate treatment decisions and can cause precious human life. A better solution is to provide an extra layer in between a conventional gateway and a remote cloud server. The extra layer denoted as fog layer helps diminishing the volume of transmitted data for guaranteeing. QoS, and saving network bandwidth by preprocessing data. At the same times, fog computing offers advanced services at the edge of the network and reduces the burden of cloud.

%
% ----------
\subsubsection{Research Objective}

% Explain the purpose of this research and the objectives of this study


%
%----------------------------------------------------
\subsection{The Contributions}


%
%----------------------------------------------------
\subsection{Thesis Organization}


%
%==========================
\section{Survey}

% Overview of the Research in the field and Taxonomy

%
%----------------------------------------------------
\subsection{Survey 1}

% Brief Overview:

% Pros

% Cons

% Comparative Analysis with our work


%
%----------------------------------------------------
\subsection{Survey 2}

% Brief Overview:

% Pros

% Cons

% Comparative Analysis with our work


%
%----------------------------------------------------
\subsection{Survey 3}

% Brief Overview:

% Pros

% Cons

% Comparative Analysis with our work


%
%----------------------------------------------------
\subsection{Survey 3}

% Brief Overview:

% Pros

% Cons

% Comparative Analysis with our work


%
%----------------------------------------------------
\subsection{Summary of the Surveyed Research Work}



%
%==========================
\section{System Model and Evaluation Metrics}

%
%----------------------------------------------------
\subsection{Scope of the Research}

% In this section, we describe the scope and limitations of the research

%
%----------------------------------------------------
\subsection{Frequently Used Terms}

%
%----------------------------------------------------
\subsection{System Model}

% Articulate the methamatical model of the system 

%
%----------------------------------------------------
\subsection{Evaluation Metrics}

% Define terms like accuracy, error, convergence, correctness, etc.

%
%----------------------------------------------------
\subsection{Simulation Model}

% Define simulation parameteres and asumptions made

%
%----------------------------------------------------
\subsection{Simulator Design Architecture}


%
%----------------------------------------------------
\subsection{Simulation Inputs and Desired Output}


%
%==========================
\section{Proposed Protocol}

%
%----------------------------------------------------
\subsection{Describe the Protocol}

%
%----------------------------------------------------
\subsection{Present the Pseudo Code of the Protocol}

%
%----------------------------------------------------
\subsection{Describe the State Diagram of the Protocol}

%
%----------------------------------------------------
\subsection{Describe the Activity Diagram of the Protocol}


%
%----------------------------------------------------
\subsection{Critically analyze the Protocol}

% Describe the Pros, Cons and the Future work of the propsoed protocol


%
%==========================
\section{Simulation and Results}


%
%==========================
\section{Conclusion}
 
%
%==========================

%\bibliographystyle{IEEEtran}
\bibliographystyle{splncs}
\footnotesize{
%\bibliographystyle{plainnat}
%\bibliography{RP2016_MIS_v1}
}
%============================================================
%
\end{document}